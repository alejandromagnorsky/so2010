\documentclass[a4paper,10pt]{article}
\usepackage[utf8]{inputenc}
\usepackage[spanish]{babel}
\usepackage[pdftex]{graphicx}

\makeindex

%opening
\title{\underline{\textbf{SuciOS}} \\ \small Standard Command-line Interface Operating System}
\author{
Magnorsky, Alejandro Ezequiel. (Leg. 50272). \\
Merchante, Mariano. (Leg. 50094).\\
Mata Suárez, Andrés Ricardo. (Leg. 50143). \\
Ballesty, Pablo Andrés. (Leg. 49359).\\
Pose, Jimena Belén. (Leg. 49015).\\
Lezica, Santiago. (Leg. 49147).\\
}
\date{Lunes 1 de Noviembre}

\begin{document}

\maketitle

\newpage
\tableofcontents

\newpage
\section{Introducción}
\subsection{Objetivo}
Continuar el Multitasker del Trabajo Práctico 2, agregándole manejo de disco,
basado en el sistema de manejo de disco de MINIX.


\subsection{Enunciado}
El trabajo consta de la realización de un manejador de disco símil al de MINIX,
para el sistema multitasker creado en el TP2. El mismo deberá poder leer y
escribir bytes a disco, pero no requerirá un filesystem. La información escrita
a disco, naturalmente debe poder persistir cuando la máquina se apague o se
reinicie.

\subsection{Programas de prueba}
Además del sistema a disco, los alumnos deberán desarrollar en assembler un
programa pequeño que carezca de stack y sea independiente de contexto. Los
alumnos deberán poner este programa en el disco en una zona conocida y agregar
un comando usuario que acceda a esta zona conocida de disco, levante
el programa en una página nueva y cree un nuevo proceso que ejecute ese programa. 
Como este programa es independiente del sistema operativo, deberían
poder ejecutarlo protegiendo el sistema operativo completo mientras el mismo
se ejecuta y deberían poder proteger todas sus zonas (incluyendo código) cuando
otro programa se ejecuta.


\subsection{Material a entregar}
Cada grupo deberá entregar los fuentes, una imagen booteable con el bootloader
y el multitasker y los datos pertinentes cargados a disco. Además deberán
entregar un informe impreso detallando las decisiones respecto a los items que
quedaron a elección del grupo, problemas presentados y la solución de los mismos
durante la realización del trabajo.

\subsection{Integrantes}
El trabajo puede hacerse en grupos de seis integrantes. Se evaluará la funcionalidad
de la aplicación así como el estilo del código y la calidad del informe entregado.

\subsection{Consideraciones}
Todo punto no explícito en este documento podrá ser interpretado a convenieniencia
del alumno, siempre dentro de los márgenes del sentido común. Ante la
duda, consultar a los docentes o enviar un mail al mail de la cátedra.

\subsection{Fecha de entrega}
La fecha de entrega del trabajo práctico es el Lunes 29 de Noviembre a las 16:00.



\newpage


\section{Desarrollo}

\subsection{Problemas encontrados y sus respectivas soluciones}

\subsection{Programas de prueba}

  \subsubsection{File system}

\end{document}
