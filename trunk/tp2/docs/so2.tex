\documentclass[a4paper,10pt]{article}
\usepackage[utf8]{inputenc}
\usepackage[spanish]{babel}
\usepackage[pdftex]{graphicx}

\makeindex

%opening
\title{\underline{\textbf{Trabajo práctico 2: Multitasker}}}
\author{
Lezica, Santiago. (Leg. 49147).\\
Ballesty, Pablo Andrés. (Leg. 49359).\\
Pose, Jimena Belén. (Leg. 49015).\\
}
\date{}

\begin{document}

\maketitle

\newpage
\tableofcontents

\newpage
\section{Introducción}
\subsection{Objetivo}
Realizar un programa que conmute diferentes procesos, asignándoles a cada 
uno un tiempo de ejecución. El multitasker no corre sobre ningún Sistema 
Operativo, se ubica en memoria utilizando un bootloader GRUB.


\subsection{Enunciado}
El trabajo consta de la realización de un Multitasker, cuyo objetivo es asignar
tiempo de ejecución a diferentes procesos en memoria. El sistema deberá ser
implementado para plataformas Intel de 32 bits, utilizando el procesador en
modo protegido. El multitasker debe ser preemptivo, es decir, cualquier tarea
puede ser desalojada del microprocesador. El encargado de administrar el CPU
es el scheduler el cual tomará como base de tiempo la interrupción de hardware
INT8 correspondiente al timer tick, para realizar la asignación de tiempo ( time
slot ).

\subsection{Contexto de tareas}
Cada grupo deberá elegir la forma en resguardar el contexto de cada tarea, las
opciones son: utilizar los TSS que provee el microprocesador Intel 386 o una
implementación propia de código.

\subsection{Scheduler}
El Multitasker deberá implementar 2 (dos) tipos de scheduling distintos. Al
menos uno de ellos deberá considerar la prioridad de los procesos para asignar
los slots de tiempo.

\subsection{Estados de procesos}
El sistema deberá estar programado de manera que se diferencien los estados
básicos de ”Corriendo”, ”Esperando” y ”Listo”. Por otra parte, cada proceso
deberá tener un valor de prioridad entre 0 y 4 que indique la importancia del
proceso. Además se deberá demostrar el funcionamiento de los mismos con
programas de prueba y se deberá poder corroborar el estado del proceso y el
porcentaje de procesador que esté ocupando con la ayuda del comando ”top”.
También deberá haber un comando kill que permita matar procesos en ejecución.
Tener en cuenta que kill debe matar también a todos los hijos de ese proceso.
Deberán desarrollar una system call (simil a yield) que fuerce al proceso actual
a ceder procesador.

\subsection{Terminales}
El usuario debe poder tener al menos 4 terminales distintas y alternar entre
ellas de manera similar a Linux.

\subsection{Intérprete de comandos}
El usuario debe poder ejecutar las diferentes tareas a través de comandos ingresados 
por teclado. La forma de los comandos quedan a elección del alumno.

\subsection{Procesos en background}
El sistema debe tener la posibilidad de correr los mismos procesos tanto en
foreground como en background. Para este último se deberá utilizar el caracter
"\&" al igual que en UNIX.

\subsection{Administración de memoria}
El sistema debe tener un módulo de administración de memoria mediante paginación 
para los procesos, el mismo se encargará de lo siguiente:

\begin{itemize}
 \item Cada proceso tendrá su stack propio en una página, a la cual solamente él
	tendrá acceso. Cada proceso podrá leer y escribir libremente sobre esta
	página pero no páginas de otros procesos.      
 \item Los procesos no poseerán un heap propio, ya que están corriendo sobre la
	misma zona de datos del SO.
 \item Ningún proceso deberá leer o escribir directamente ninguna variable global
	del SO. En caso de que haya variables globales que estén pensadas para
	ser leidas por procesos usuario, deberán tener una función que las copie a
	una zona de heap propia al proceso, simulando un system call.
\end{itemize}

\subsection{Proveedor de Servicios}
Deberán implementar dos procesos proveedores de servicios, que estarán corriendo 
desde la inicialización del sistema. Estos procesos estarán constantemente
bloqueados esperando pedidos. Cuando algún proceso usuario requiera algún
servicio de algún servidor, lo llamará mediante un IPC (puede ser cualquiera,
incluso sencillo). El proceso servidor correrá con máxima prioridad y no podrá
ser interrumpido por otros procesos usuario, salvo que decida ceder procesador
voluntariamente.

\subsection{Programas de prueba}
Cada grupo deberá desarrollar tareas, que funcionarán como programas de
prueba, los mismos deberán demostrar vulnerabilidades y virtudes del trabajo,
y servirán para demostrar la implementación del TP.


\newpage


\section{Desarrollo}


\newpage
\section{Conclusiones}

\bigskip
\end{document}
