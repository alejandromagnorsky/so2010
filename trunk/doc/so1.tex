\documentclass[a4paper,10pt]{article}
\usepackage[utf8]{inputenc}
\usepackage[spanish]{babel}

\makeindex

%opening
\title{\underline{\textbf{Filesystems, IPCs y Servidores Concurrentes}}}
\author{
Lezica, Santiago. (Leg. 49147).\\
Ballesty, Pablo Andrés. (Leg. 49359).\\
Pose, Jimena Belén. (Leg. 49015).\\
}
\date{}

\begin{document}

\maketitle

\newpage
\tableofcontents

\newpage
\section{Introducción}
\subsection{Objetivo}
El objetivo de este trabajo es familiarizarse con el uso de sistemas cliente-
servidor concurrentes, implementando el servidor mediante la creación de proce-
sos hijos utilizando fork() y mediante la creación de threads. Al mismo tiempo,
ejercitar el uso de los distintos tipos de primitivas de sincronización y comunicación
de procesos (IPC) y manejar con autoridad el filesystem de Linux desde
el lado usuario.

\subsection{Enunciado}
Se desea implementar un simulador de colonia de hormigas simplificado. En
el mismo habrá un hormiguero, varias hormigas y comida esparcida a lo largo
del mundo. Las hormigas deberán recolectar la comida y traerla al hormiguero.
Para ello deberán leer la información del ambiente y comunicarse entre ellas
de manera eficiente. El objetivo de la simulación es traer la mayor cantidad
de comida al hormiguero. Después de 10000 turnos, o cuando ya no haya más
comida en el mundo, la simulación finaliza.\\
El servidor leerá del archivo de configuración la información acerca del mundo,
que tendrá forma de grilla. Particularmente estará la ubicación del hormiguero,
cada pieza de comida, su tipo (simple o grande) y las hormigas en su posición inicial.
A continuación, empezará la simulación, en donde, por turnos simultáneos, to-
das las hormigas deberán realizar una acción. Esta acción podría ser o bien:

\begin{enumerate}
 \item Moverse a un casillero contiguo horizontal o vertical.
 \item Oler los casilleros vecinos para detectar rastros, hormigas o comida.
 \item Levantar una pieza de comida que esté en un casillero contiguo horizontal o vertical.
 \item Moverse a un casillero vecino dejando un rastro.
 \item Emitir un grito.
\end{enumerate}

Dos hormigas no podrán ocupar el mismo casillero y 1 hormiga no podrá ocupar 
el mismo casillero que una pieza de comida sin levantar. En caso de que al
finalizar un turno, 2 o más hormigas intenten moverse al mismo casillero, solo
una lo logrará y la otra fallará su movimiento. La unica excepción a esta regla
es el hormiguero. Pueden haber infinitas hormigas en el casillero del hormiguero.
Las hormigas pueden dejar y detectar un rastro. Este rastro es un valor decimal
entre 0 y 1, en donde 1 es un rastro recién puesto y 0 es ”no hay rastro en
absoluto”. Al avanzar y dejar rastro, el valor de rastro dejado SIEMPRE será 
de valor 1. y por cada turno el rastro decrementará en 0.01.\\

Las hormigas SIEMPRE saben la orientación del hormiguero relativa a donde
están paradas, no así la distancia. (Es decir, una hormiga puede preguntar, sin
invertir turnos en ello, hacia donde está el hormiguero y recibir como respuesta
(N,S,E,W,NE,NW,SE,SW).\\

Las hormigas tienen una memoria muy limitada y solo pueden recordar 2 posiciones 
en el mapa, una de ellas siendo siempre el hormiguero. Es decir, una
hormiga puede decidir almacenar una posición del tablero para luego preguntarse
en que dirección está.\\

Las hormigas tienen una memoria muy limitada y solo pueden recordar 2 posiciones 
en el mapa, una de ellas siendo siempre el hormiguero. Es decir, una
hormiga puede decidir almacenar una posición del tablero para luego preguntarse
en que dirección está.
Cuando una hormiga grita, todas las hormigas reciben la posición de la hormiga
que grita y la distancia hamiltoniana entre ellas. Al escuchar un grito, una
hormiga puede optar por reemplazar su memoria por la posición de la hormiga
que gritó.\\

Existen 2 tipos de comida: chica y grande. La comida chica puede ser transportada 
por una hormiga sin dificultad y tiene valor 1. La hormiga simplemente
tiene que posicionarse en un casillero contiguo y utilizar un turno para levantar
la comida. La comida grande vale 5 puntos, puede ser transportada por una
hormiga, pero necesita de 2 hormigas para ser levantada, es decir: debe haber
2 hormigas posicionadas contigua a la comida y ambas deben utilizar un turno
para levantar o asistir en levantar la comida.\\

\subsection{Actividades}
Implemente la simulación utilizando procesos y threads y haga cuatro versiones del sistema, 
usando las siguientes primitivas de IPC:
\begin{enumerate}
 \item 
    \begin{itemize}
      \item Pipes o fifos.
      \item Colas de mensajes - System V o POSIX.
      \item Memoria compartida o mmap(), Semáforos System V o POSIX.
      \item Sockets - TCP o de dominio Unix.
    \end{itemize}
  \item El archivo de configuración tendrá el siguiente formato:
    \begin{itemize}
     \item Una línea con la longitud y alto del tablero separados por coma. Ej: 6,8
      significa un tablero de 6 columnas y 8 filas.
     \item Una línea con la posición del hormiguero separada por coma, teniendo en
      cuenta que la posición superior izquierda es 0,0. Ej: 3,4 significa que el
      hormiguero está en la cuarta columna, quinta fila.
     \item Una línea con la cantidad de hormigas N. Todas las hormigas empiezan
      en el hormiguero.
     \item 

    \end{itemize}

\end{enumerate}



\bigskip
\end{document}
